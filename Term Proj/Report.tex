\documentclass{homework}
\author{Aiden Taylor \& Julia Fay}
\class{ME 419: Advanced Control Systems}
\date{\today}
\title{Solar Tracker Control Problem}
\usepackage{titlesec}
\usepackage{amsmath}
\titleformat{\section}    
       {\fontsize{14}{17}\bfseries}{\thesection}{1em}{}

\begin{document}  \maketitle
\section*{Background}

The goal of this lab is to understand the impact of accurately modeling friction within a system and how to tune a PID controller to obtain a desired system output. This will be accomplished by modeling and comparing the behavior of a solar tracking system both with and without bearing friction.

\img<panel>[0.75]{Single Axis Solar Tracking System}{panel.png}

The image above shows a single-axis solar tracking system. The main system components are the solar panel, the bearings, and the motor on the panel's central axis which needs a controller. The tracker follows the sun during the day, but is designed to return to a flat orientation as soon as an energy dip is detected in the panel's power output– whether it be from horizon shading, clouds, or other factors. This allows the panel to accumulate as much ambient light as possible to continue producing the maximum possible amount of power even when the sun is not directly hitting it. Your goal is to model the system provided using concepts from ME322 to obtain a system transfer function, then design a controller that will give the fastest possible response when going from an angled position, to horizontal while also fulfilling the provided expectations for settling time and damping. The controller will be tuned for two cases: including bearing friction, and ignoring bearing friction. 

\newpage
\section*{System Modeling}
\section*{Obtaining the System Transfer Function}

\question The main system components include the panel, and the motor. The panel can be modeled as a mass rotating about its centroidal axis with inertia J. Include viscous damping b. The motor is a smaller system that converts an input voltage to rotational motion. The motor circuit should have the motor inductance L and resistance R. The rotational inertia J of the motor should be in parallel with the panel inertia. See Figure \ref{wheel}.

\img<wheel>[0.25]{Full System Model}{model.png}

\subsection*{System Constants}
\begin{align*}
	b & = 10         &  & \text{Viscous Damping Coefficient}         \\
	R & = 1           &  & \text{Winding Resistance}         \\
	L & = 0.078*10^{-3}           &  & \text{Winding Inductance}      \\
	J & = 5        &  & \text{Total rotational intertia}         \\
	K_t & = 5           &  & \text{Motor Torque Constant}         \\
	K_v & = 5          &  & \text{Motor Voltage Constant}      \\
	K_c & = 2/(\pi) &  & \text{Conmpensation Constant}	\\
\end{align*}

\question From this system schematic, create a linear graph and normal tree to obtain the elemental and constraint equations. Using these equations, create a symbolic block diagram that models the system starting with the input voltage, and ending with the panel's angular position.

\question Next, calculate the system transfer function in symbolic and numerical form. The system should be modeled as a second order system.

\section*{Designing the Controller with Pole Placement for the Frictionless System}

\question Calculate the A, B, C matrices of the system in phase variable form where the output is the angular position of the panel.  

\question The controller should be designed such that the system stays within 10\% OS and has a settling time of 2 seconds. Calculate the K values using these requirements with viscous damping is set to zero. Calculate the Anew matrix with your K values using the following equation: 

\[ A_{new} = A_{old} - BK \]

\newpage
\section*{Modeling Frictionless System Behavior}
\question Enter your A\textsubscript{new}, B, C, and D matrices for the frictionless system into MatLab and use the ss function to create a system with your matrices. Note: Your D matrix will be [0].

\question Use the step function to generate a plot of the system's angular position with respect to time. Note: the desired movement of the system of 90 degrees is accounted for in K\textsubscript{c}.

\question Comment on the system's damping, and if the controller accurately produced the desired results. 

\section*{Modeling the System Behavior with Friction}
\question Change the damping coefficient to the value provided in the system values section while keeping the K values designed for the frictionless system. Plot and analyze the positional output graph to note if the simulation successfully reached the parameters of staying within 10\% overshoot and having a settling time of 2 seconds when friction is added. What kind of damping does the system exhibit?

\question Recalculate the A, B, and C matrices and redesign the K values such that the new system meets the requirements. 

\question Plot the new system step response. What kind of damping do you see now? 

\question Comment on the importance of correctly modeling friction within a system and the potential issues one may run into if it is unaccounted for. 

\section*{Deliverables}

\begin{itemize}
  \item System modeling materials, including the system schematic, linear graph, and normal tree. 
  \item System block diagram and transfer function derivation 
  \item Plots of the frictionless system response, frictional system with frictionless controller response, and frictional system with new controller response
  \item Answers to all questions posed in the procedure
\end{itemize}
\newpage
\section*{Solution}
\img<lgnt>[0.75]{Linear Graph and Normal Tree}{lgnt.png}
\begin{align*}
	& \text{Elementals} &  & \text{Constraints}	\\
	& V_R = Ri_R         &  & i_R = i_L        \\
	& \frac{1}{L}\int{}V_Ldt         &  & V_L = V_{in} - V_R - V_1        \\
	& V_i  = K_v\omega_2       &  & \omega_2 = \omega_J        \\
	& \tau_2 = -K_vi_1         &  & i_1 = i_L        \\
	& \omega_J = \frac{1}{J}\int{}\tau_Jdt         &  & \tau_J = -\tau_2 - \tau_b        \\
	& \tau_b = b\omega_b         &  & \omega_b = \omega_J        \\
\end{align*}

\img<bd>[0.75]{Block Diagram}{bd.png}
\newpage
\subsection*{Transfer Function}

\[\frac{\theta_J}{\theta_{in}} = \frac{\frac{K_vK_c}{RJ}}{s^2 + \frac{Rb + K_v^2}{RJ}s}\]

\subsection*{A, B, C Matrices}

These are the expected A, B, C Matrices for the frictionless system
\[\]
$A = \begin{bmatrix}
0 & 1\\
0 & -(\frac{K_v^2}{RJ})
\end{bmatrix}\begin{bmatrix}
x_1\\
x_2
\end{bmatrix}$
\[\]
$B = \begin{bmatrix}
0\\
1
\end{bmatrix}\theta_{in}$
\[\]
$C = \begin{bmatrix}
\frac{K_vK_c}{RJ} & 0
\end{bmatrix}\begin{bmatrix}
x_1\\
x_2\end{bmatrix}$

\subsection*{A\textsubscript{new} and K matrices}

These are the expected $A_{new}$ and K Matrices for the frictionless system:
\[\]
$A_{new} = \begin{bmatrix}
0 & 1\\
-11.42 & -54
\end{bmatrix}$
\\
$K = \begin{bmatrix}
11.42 & -121
\end{bmatrix}$
\subsection*{A, B, C Matrices}
These are the expected A, B, C Matrices for the system with friction:
\[\]
$A = \begin{bmatrix}
0 & 1\\
0 & -(\frac{Rb + K_v^2}{RJ})
\end{bmatrix}\begin{bmatrix}
x_1\\
x_2
\end{bmatrix}$
\\
$B = \begin{bmatrix}
0\\
1
\end{bmatrix}\theta_{in}$
\\
$C = \begin{bmatrix}
\frac{K_vK_c}{RJ} & 0
\end{bmatrix}\begin{bmatrix}
x_1\\
x_2
\end{bmatrix}
$

\subsection*{A\textsubscript{new} and K matrices}

These are the expected $A_{new}$ and K Matrices for the system with friction:
\[\]
$A_{new} = \begin{bmatrix}
0 & 1\\
-11.42 & -4
\end{bmatrix}\begin{bmatrix}
x_1\\
x_2
\end{bmatrix}$
\\
$K = \begin{bmatrix}
11.42 & -171
\end{bmatrix}$
\newpage

\section*{Rubric}

\begin{center}
\begin{tabular}{ | l | l | l | l | l |}
\hline
Criteria & Full Marks & Partial Credit & Points\\ \hline
Obtaining the System & (2) System Schematic & Partial Credit is awarded & 8 \\Transfer Function&(2) Linear Graph/Normal Tree &  for a correct process with & \\&(2) Block Diagram &  an incorrect final answer. & \\&(2) Transfer Function &  (0-8 pts) &
 \\ \hline
Designing the Controller & (2) A,B,C, and D matrices  & Partial Credit is awarded & 6\\With Pole Placement for & for frictionless system &for a correct process with&\\the Frictionless System & (1) K values & an incorrect final answer & \\ & (1) Anew matrix for & (0-4 pts) & \\ & frictionless system & &\\ \hline


Modeling the System & (1) Plot for the frictionless system& Partial Credit is awarded & 8\\Behavior with Friction &  (1) Friction Plot with wrong K & for a correct process with & \\values calculated for the & (2) K values for frictional system & an incorrect final answer. & \\frictionless system. & (2) Anew matrix for frictional system & Missing or incorrect plots & \\&(1) Plot showing the system with & will receive zero credit. & \\&friction applied with corresponding & (0-8 pts)&\\ &K values & & \\&(1) Response to discussion question & &\\
\hline
 &  & Total & 20 \\ \hline
\end{tabular}
\end{center}
\newpage

\section*{Reflections}
\subsection*{Julia's Reflection}
While writing this problem, I learned about the process of designing a system that will produce a specific outcome intended to teach a desired concept. In our case, we wanted to teach students about the importance of correctly modeling friction within a system. This is important because incorrectly doing so can result in a controller that does not produce the needed behaviors. Modeling this system was straightforward as it was very similar to the DC motor model we have covered as ME students. However, I had to tweak the system constants until the system had an overdamped response when friction was applied while still using the K values calculated for the frictionless system. To accomplish this, I had to increase the friction constant and decrease the motor voltage constant. The drawback of this process is that the resulting constants may not be realistic values you would find from a motor spec sheet or a friction analysis. However, for this mini-lab, getting the friction concept across was prioritized over providing a realistic system. Overall, it was enjoyable to think about education from the teachers perspective. When designing labs in particular, the designer must balance providing a real world experience with an educational one. 

\subsection*{Aiden's Reflection}
This assignment taught me about how to take concepts that I have learned and try and combine them into a new problem. This problem was based on our senior project, which requires a controller to turn the panel to a horizontal position when panel shading occurs. I took this idea and simplified it into a DC motor with a large rotating mass attached. Although it does not exactly emulate our project, it still shows the student what type of control this system would need. Also, this assignment was very informative in terms of how to use LaTex. This was my first time using this software, and I can now bring these skills with me into the future. 

\section*{MatLab Solution On Next Page}

\end{document}